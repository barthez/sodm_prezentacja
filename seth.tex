\section{SETH - wspomaganie decyzji w toksykologi}

\begin{frame}
\frametitle{SETH - system wspomagania decyzji\\ \hskip8ex w toksykologi}
    
\begin{itemize}
    \item  Celem SETH jest udzielać konkretnych porad dotyczących leczenia i
    monitorowania zatrucia lekami.
    \item Wprowadzony w 1992 roku.
    \item Baza danych zawiera 1153 najbardziej toksyczne leki z 78 klas
    toksykologicznych.
    \item Używany jako wsparcie dla pracowników telefonicznej pomocy w
    przypdku zatruć.
\end{itemize}

\end{frame}

\begin{frame}[allowframebreaks]
\frametitle{SETH - podejmowanie decyzji}

System SETH symuluje działanie ekspertów.

\begin{enumerate}
    \item Wyznaczenie dawki, substancji i czasu od przyjęcia leku. W znacznej 
    większości przypadków informacje te mogą być określone przez pacjenta 
    lub rodzinę na podstawie prostego zestawu pytań.
    \item Wnioskowanie objawów jest proste w przypadku zażycia jednego
    toksycznego leku (dokładnie opisane reakcje organizmu).
\framebreak
    \item Wnioskowane podczas zatrucia róznymi substancjami równoczesnie staje
    się o wiele trudniejsze.
    \begin{itemize}
        \item Zasięgnięcie pomocy w Poison Control Center (Centrum kontroli
        Zatruć)
        \item Należy stwierdzić czy dawka każdej z substancji jest już
        toksyczna czy jeszcze nie.
        \item Objawy kliniczne są bardzo ważne w toksykologi, dlatego brak
        objawów pozwala na wykluczenie substancji.
        \item Po podjęciu leczenie w stosunku do wybranych substancji, należy
        równiez pamietać o substancjach pominiętych w trakcie późniejszego
        leczenia i kontoli zatrucia.
        \item W przypadku nie znanych leków klasa toksykologiczności jest
        określana na podstawie objawów.
    \end{itemize}

\end{enumerate}

\end{frame}

\begin{frame}
\frametitle{SETH - Baza wiedzy}

\begin{itemize}
    \item Baza wiedzy składa się z warunkow, obiektów, wniosków, zasad i opisów
    przypadków.
    \item Model konsultacji składa się z ustaleń, hipotez i reguł decyzyjnych.
    \item Wnioski stawiane przez system zawierają zalecenia leczenia i kontroli
    pacjenta.
    \item Baza danych zawiera informacje nt. leków, klas toksykologicznych,
    potencjalnych klasyfikacji zatruć oraz sposobów leczenia zaleznych od
    stopnia zatrucia.
\end{itemize}


\end{frame}

