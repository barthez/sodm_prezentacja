\section{ACORN - ostra niedokrwienna choroba serca}

\begin{frame}
\frametitle{ACORN - ostra niedokrwienna choroba serca}

\begin{description}  
\item[Admit to the Ccu OR Not] (ang. przyjąć na oddział intensywnej terapii, czy nie) System ekspertowy wspomagający podejmowanie decyzji w przypadkach bólu klatki piersiowej u pacjentów na izbie przyjęć. System wspomaga pielęgniarki w szybszym stwierdzaniu stanu i obsłudze pacjentów z ostrą niedokrwienną chorobą serca.
\end{description}
\end{frame}

\begin{frame}
\frametitle{ACORN - budowa}

Hybrydowy system łączący w sobie:
\begin{itemize}
  \item wnioskowanie wstecz
	\item system oparty o wiedzę
	\item sieci Bayes'a
\end{itemize}

\end{frame}

\begin{frame}
\frametitle{ACORN - skuteczność}
Kontrole medyczne stwierdziły, że wśród przypadków z ostrą niedokrwienną chorobą serca 38\% pacjentów było odsyłanych do domów w wyniku błędnej diagnozy, a czas przyjęcia pozostałych wynosił średnio 115 minut.
Po wprowadzeniu systemu ACORN w 1987 roku w Szpitalu Westminster w Londynie stwierdzono spadek błędnych diagnoz do 20\% oraz skrócenie średniego czasu obsługi o 20 minut.

\end{frame}

\begin{frame}
\frametitle{ACORN - krytyka}

Ze względu na małą próbę statystyczną oraz wykazaną niedbałość prowadzenia statystyk użycia systemu nie można jednoznacznie potwierdzić prawdziwości przedstawionych wyników.\\
Podejmowano próby ulepszenia systemu, do roku 1990 stwierdzono poprawę diagnostyki schorzenia, jednak nie udało się jednoznacznie rozstrzygnąć, jak duży udział w sukcesie miał nowatorski system, a ile zawdzięczano tzw. efektowi Hawthorne polegającemu na świadomości użytkowników systemu w uczestnictwie w eksperymencie i bycia pod ciągłą obserwacją, które powodowały zmiany w zachowaniu i zwiększenie jakości wykonywanej pracy.

\end{frame}

