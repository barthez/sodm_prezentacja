\section{ACORN}

\begin{frame}
\frametitle{ACORN}
    
ACORN (\textit{Admit to the Ccu OR Not})

Hybrid rule-based \& Bayesian system for advising on management of chest pain patients in the emergency room.

Accident \& Emergency Department Westminster Hospital, London
DATE COMMISSIONED: 1987

Medical audits determined that 38\% of patients attending with acute ischaemic heart disease were sent home in error and the median time till CCU admission for the remainder was 115 minutes. ACORN, a hybrid backward- chaining rule-based and Bayesian system, was built for use by senior nurses to assist in the management of these patients. In a randomised controlled trial on 150 patients in 1987 the false negative rate in both control & ACORN cases fell to 20\%; this may have been a carryover or Hawthorne effect. This trial also identified significant problems with ACORN, and it was subsequently revised and appeared to be effective at reducing the median time to CCU admission by 20 minutes, though this was an uncontrolled study.
The system was in routine use at Westminster during 1987-90. In 1990: the mean usage rate per eligible case claimed by the 15 users was 77\%, but when unequivocal evidence of use was looked for in the patient records, this was present in only 23\% of eligible cases. There were approx. 15 eligible cases per week = 750 per year, so this scales up to between 175 and 580 uses of ACORN per annum.


ACORN
Systemem ekspertowym, który wspomaga lekarzy w przypadkach nagłych,
objawiających się u pacjentów bólem klatki piersiowej. Kontrole medyczne w 38\%
błędnie decydowały o wysyłaniu pacjentów, z ostrą niedokrwienną chorobą serca, do
domu. Średni czas przyjęcia na oddział kardiologiczny wynosił 115 minut. ACORN
został stworzony, aby wspomóc lekarzy i pielęgniarki w takich przypadkach. System
był używany na oddziale nagłych wypadków w szpitalu Westminster w Londynie, w
latach 1987-1990. W 1990 średnie użycie programu w 15 odpowiednich przypadkach
wynosiło 77\%, jednak jednoznacznych dowodów szukano w dokumentacji
medycznej, która wskazywała na jedyne 23\% użycia programu. Przeciętnie 15
przypadków tygodniowo, czyli 750 przypadków rocznie, użycie programu ACORN
wahało się między 175 a 580x rocznie. Dzięki próbie kontrolnej na 150 pacjentach w
1987 roku, wykryto ważne niedociągnięcia programu ACORN, który był następnie
korygowany. Okazał się skuteczny w obniżeniu średniego czasu przyjęcia pacjenta na
oddział kardiologiczny o 20 minut.


\end{frame}
