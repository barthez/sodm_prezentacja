\section{CCIS - rak szyjki macicy}

\begin{frame}
\frametitle{CCIS}

  \begin{itemize}
    \item \textit{Cervical Cancer Information System}
    \item System stworzony do monitorowania wpływu i efektywności programu wykrywania raka szyjki macicy.
  \end{itemize}

\end{frame}

\begin{frame}
\frametitle{CCIS - Viva Mulher}

  \begin{itemize}
    \item Stworzony aby wspomagać narodowy program wykrywania raka szyjki macicy w Brazylii znany jako Viva Mulher.
    \item Został wprowadzony w 1997 roku, od tamtej pory używany jest z ogromnym sukcesem.
    \item System jest w stanie radzić sobie z przepływem informacji między laboratoriami a jednostkami medycznymi, z pierwszego, drugiego i trzeciego poziomu. W tym samym czasie jest w stanie tworzyć raporty oceniające program badawczy.
  \end{itemize}

\end{frame}

\begin{frame}
\frametitle{CCIS - twórcy}

Powstaw we współpracy:
  \begin{itemize}
    \item Narodowego Istytutu Badań nad Rakiem w Brazylii (INCA),
    \item Cancer Care International (CCI),
    \item Tecso, firmy zajmującej się rozwojem oprogramowania komputerowego.
  \end{itemize}

\end{frame}

\begin{frame}
\frametitle{CCIS - zadania}

Pomimo wielu ról systemu: m.in. rekrutowanie ludności docelowej oraz masowa identyfikacja ludności, zbieranie informacji dotyczących demografii i badań oraz raportowanie wyników do kobiet i lekarzy, system zapewnia również link pomiędzy diagnozą, badaniami kontrolnymi oraz leczeniem, monitoruje również jakość programu i posiada mechanizmy kontrolne.

Do najważniejszych zadań systemu należą:
  \begin{itemize}
  \item Recruit the target population and identify under-screened and over-screened population;
  \item Collect demographics and screening event information, and report results to women and health care providers;
  \item Provide links between diagnosis;
  \item Recall, follow-up, and treatment events;
  \item Monitor programs quality and provide control mechanisms.
  \end{itemize}
\end{frame}




