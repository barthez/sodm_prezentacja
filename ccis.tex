\section{CCIS - rak szyjki macicy}

\begin{frame}
\frametitle{CCIS}

  \begin{itemize}
    \item \textit{Cervical Cancer Information System}
    \item System stworzony do monitorowania wpływu i efektywności programu wykrywania raka szyjki macicy.
  \end{itemize}

\end{frame}

\begin{frame}
\frametitle{CCIS - Viva Mulher}

  \begin{itemize}
    \item Stworzony aby wspomagać narodowy program wykrywania raka szyjki macicy w Brazylii znany jako Viva Mulher.
    \item Został wprowadzony w 1997 roku, od tamtej pory używany jest z ogromnym sukcesem.
    \item System jest w stanie radzić sobie z przepływem informacji między laboratoriami a jednostkami medycznymi, z pierwszego, drugiego i trzeciego poziomu. W tym samym czasie jest w stanie tworzyć raporty oceniające program badawczy.
  \end{itemize}

\end{frame}

\begin{frame}
\frametitle{CCIS - twórcy}

Powstał we współpracy:
  \begin{itemize}
    \item Narodowego Instytutu Badań nad Rakiem w Brazylii (INCA),
    \item Cancer Care International (CCI),
    \item Tecso, firmy zajmującej się rozwojem oprogramowania komputerowego.
  \end{itemize}

\end{frame}

\begin{frame}
\frametitle{CCIS - zadania}

Do najważniejszych zadań systemu należą:
  \begin{itemize}
  \item Angażowanie ludności oraz identyfikacja stopnia objęcia populacji badaniami przesiewowymi.
  \item Pozyskiwanie danych demograficznych oraz raportowanie wyników do kobiet i służb medycznych.
  \item Przypominanie, wznawianie i kontynuacja leczenia  oraz badań profilaktycznych.
  \item Monitorowanie jakości programu i zapewnianie kontroli nad mechanizmami.
  \end{itemize}
\end{frame}
