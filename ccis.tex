\section{CCIS}

\begin{frame}
\frametitle{CCIS}
  Cervical Cancer Information System -   
 has been developed to monitor the impact and effectiveness of a cervical cancer screening program. 

The system is capable of handling the flow of information between laboratories and health care units from primary, secondary and tertiary levels; and to generate evaluation reports of the screening program. Among the systems many roles are:

Recruit the target population and identify under-screened and over-screened population;
Collect demographics and screening event information, and report results to women and health care providers;
Provide links between diagnosis;
Recall, follow-up, and treatment events;
Monitor programs quality and provide control mechanisms.
CCIS has been developed to support the National Cervical Cancer Screening Program for Brazil, which is called Viva Mulher. It has been used with great success since its installation in 1997.


Availability: 

CCIS for Windows (CD-ROM), further information: http://www.tecso.com.br


Author: 

CCIS has been developed by the National Cancer Institute of Brazil (INCA), the Cancer Care International (CCI), and Tecso, a software-developing company.


CCIS (System Informacyjny dotyczący nowotworu szyjki macicy)
Powstał w celu monitorowania wpływu i efektywności programu zajmującego
się badaniem nowotworu szyjki macicy. System jest w stanie radzić sobie z
przepływem informacji między laboratoriami a jednostkami medycznymi, z
pierwszego, drugiego i trzeciego poziomu; w tym samym czasie jest w stanie tworzyć
raporty oceniające program badawczy. Pomimo wielu ról systemu: m.in.
rekrutowanie ludności docelowej oraz masowa identyfikacja ludności, zbieranie
informacji dotyczących demografii i badań oraz raportowanie wyników do kobiet i
lekarzy, system zapewnia również link pomiędzy diagnozą, badaniami kontrolnymi
oraz leczeniem, monitoruje również jakość programu i posiada
mechanizmy
kontrolne.
5
CCIS został stworzony aby wspierać Narodowy Program Badań nad Rakiem w
Brazylii, zwanym Viva Mulher. Jest stosowany z powodzeniem od 1997 roku.


\end{frame}
