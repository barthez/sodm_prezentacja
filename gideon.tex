\section{Gideon - diagnostyka chorób tropikalnych i zakaźnych}

\begin{frame}
\frametitle{Gideon - diagnostyka chorób\\
 \hskip8ex tropikalnych i zakaźnych}

System wspomamagania decyzji w dignostyce i leczeniu chorób tropikalnych i
zakaźnych. Potrafi diagnozować choroby w oparciu o symptomy, oznaki, testy
laboratoryjne oraz profil dermatologiczny. Szczególnym aspektem chorób
zakaźnych w tym systemie jest miejsce jej pochodzenia.

Aplikacja dostępna online pod adresem: \url{http://www.gideononline.com/}

\end{frame}

\begin{frame}[allowframebreaks]
 \frametitle{Moduły systemu Gideon}

System Gideon działa w oparciu o 4 moduły:

\begin{description}
 \item[Diagnoza] - działa w oparciu o wyniki badania podmiotowego i przedmiotowego
 (data zachorowania, czas inkubacji, objawy, symptomy czy dane laboratoryjne).
 Wyjściem modułu jest macierz bayesowska porównująca objawy ze znanymi
 jednostami chorobowymi. Moduł pozwala na aktywną współpracę z lekarzem w celu
 wprowadzenia nowych danych lub upewnienia się dlaczego system dał taką
 odpowiedź. Pozwala również na symulacje przypadku gdy podane są dodatkowe
 kraje tranzytowe (przydatne w przypadku uciekinierów czy nielegalnych
 emigrantów). Możliwa jest również symulacja ataku bioterrorystycznego.
 \framebreak
 \item[Epidemiologia] - jest to aktualizowana na bieżąco baza danych
 epidemologicznych o ponad 230 chorobach. Dane odnoszą się do krajów, czynników
 przenoszących oraz uogulnionych syndromów. Moduł umożliwa skomplikowane
 analizy porównawcze.
 \item[Terapia] - opisuje farmakologię i zastosowanie leków. Zawiera informacje
 o przeszło 2500 nazw leków, proponowane dawkowanie dla dzieci i dorosłych oraz
 informacje o interakcjach i toksyczności. Baza szczepionek podaje zalecenia
 terminów szczepień i okres ich ważności dla poszczególnych krajów.
 \framebreak
 \item[Mikrobiologia] - pozwala na identyfikację bakterii, mykobakterii i
 drożdży na podstawie opisanej reakcji fenotypu. Interakcja z użytkownikiem
 przebiega analogicznie do modułu \textbf{Diagnoza}.
 

 \end{description}

\end{frame}


