
\begin{frame}
\frametitle{Wstęp}

Zadaniem dedykowanych systemów medycznych jest:
\begin{itemize}
 \item Wsparcie diagnostyki medycznej dla lekarzy.
 \item Wykrywanie związków przyczynowo-skutkowych (objawowo-chorobowych).
 \item Umożliwienie pacjentom wstępnej diagnostyki bez udziału lekarza.
\end{itemize}

\end{frame}

\begin{frame}
\frametitle{Budowa systemu}

Systemy w medycynie często oparte są o systemy ekspertowe, na które składają się:
\begin{enumerate}
\item Cechy kliniczne pacjentów dostarczają danych wejściowych do obliczeń, których wyniki są niezbędne do podejmowania decyzji.
\item Do ich oceny potrzebna jest wiedza w postaci bazy danych.
\item Niezawodność wnioskowania zależy zatem od:
  \begin{itemize}
    \item jakości bazy wiedzy,
    \item skuteczności maszyn wnioskowania,
    \item dokładności danych wejściowych, czyli cech klinicznych.
  \end{itemize}
\end{enumerate}
\end{frame}
