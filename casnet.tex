\section{CASNET/Glaucoma - diagnoza i leczenie jaskry}

\begin{frame}
\frametitle{CASNET/Glaucoma - diagnoza \\
    \hskip8ex i leczenie jaskry}


\begin{itemize}
 \item Oparty o system CASNET (Casual ASsociation NETworks) opracowany w 1960 roku.
 \item Rozpoznaje i proponuje leczenie dla stanów chorobowych związanych z
 jaskrą.
 \item Wybór specjalistycznego leczenia oparty jest o indywidualne obserwacje i
 wnioski.
\end{itemize}

\end{frame}


\begin{frame}
\frametitle{Struktura wnioskowania}

Wszystkie fakty w CASNET przechowywane są w czterech płaszczyznach:

\begin{enumerate}
 \item Obserwacje
 \item Stany chorobowe (sieć możliwych przebiegów choroby)
 \item Kategorie chorób
 \item Sposoby leczenia
\end{enumerate}

\end{frame}

\begin{frame}[allowframebreaks]
 \frametitle{Powiązania między płaszczyznami\\
 \hskip8ex wnioskowania}

\begin{description}
 \item[Połaczenie obserwacji i stanów chorobowych] - każde połaczene objawu ze
 stanem chorobowym opisane jest i współczynnikiem stopienia zaufania, którego 
 wartość dodatnia sugeruje wysoką korelację objawu i choroby, a wartość ujemna 
 niską, lub brak korelacji.
 \item[Połaczenie stanów chorobowych z klasami chorób] - połaczenie wiąże
 przebieg choroby z odpowiadającą jej klasą. Przebieg choroby zawierający pełną
 ścieżkę stanów (od początkowego do końcowego) przypisany jest do specyficznej
 klasy choroby. Przebieg który zakończył się na stanie pośrednim może być
 połaczony z niedokładnie określoną klasą.
 \framebreak
 \item[Połaczenie klasy choroby z leczeniem] - każda klasa ma przypisany do
 niej odpowiedni sposób lecznia.
 \item[Połączenie objawów z leczeniem] - choć to połaczone zaburza spójnośc
 schematu, pozwala na podjęcie szybszych działań (np. dodatkowych badań) w
 przypadku gdy objawy jasno sugerują sposób leczenia.

\end{description}

\end{frame}
